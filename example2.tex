% Options for packages loaded elsewhere
\PassOptionsToPackage{unicode}{hyperref}
\PassOptionsToPackage{hyphens}{url}
\PassOptionsToPackage{dvipsnames,svgnames,x11names}{xcolor}
%
\documentclass[
]{article}

\usepackage{amsmath,amssymb}
\usepackage{iftex}
\ifPDFTeX
  \usepackage[T1]{fontenc}
  \usepackage[utf8]{inputenc}
  \usepackage{textcomp} % provide euro and other symbols
\else % if luatex or xetex
  \usepackage{unicode-math}
  \defaultfontfeatures{Scale=MatchLowercase}
  \defaultfontfeatures[\rmfamily]{Ligatures=TeX,Scale=1}
\fi
\usepackage{lmodern}
\ifPDFTeX\else  
    % xetex/luatex font selection
\fi
% Use upquote if available, for straight quotes in verbatim environments
\IfFileExists{upquote.sty}{\usepackage{upquote}}{}
\IfFileExists{microtype.sty}{% use microtype if available
  \usepackage[]{microtype}
  \UseMicrotypeSet[protrusion]{basicmath} % disable protrusion for tt fonts
}{}
\makeatletter
\@ifundefined{KOMAClassName}{% if non-KOMA class
  \IfFileExists{parskip.sty}{%
    \usepackage{parskip}
  }{% else
    \setlength{\parindent}{0pt}
    \setlength{\parskip}{6pt plus 2pt minus 1pt}}
}{% if KOMA class
  \KOMAoptions{parskip=half}}
\makeatother
\usepackage{xcolor}
\usepackage[top=1.5cm,bottom=1.5cm,left=0.2cm,right=0.2cm,landscape]{geometry}
\setlength{\emergencystretch}{3em} % prevent overfull lines
\setcounter{secnumdepth}{-\maxdimen} % remove section numbering
% Make \paragraph and \subparagraph free-standing
\makeatletter
\ifx\paragraph\undefined\else
  \let\oldparagraph\paragraph
  \renewcommand{\paragraph}{
    \@ifstar
      \xxxParagraphStar
      \xxxParagraphNoStar
  }
  \newcommand{\xxxParagraphStar}[1]{\oldparagraph*{#1}\mbox{}}
  \newcommand{\xxxParagraphNoStar}[1]{\oldparagraph{#1}\mbox{}}
\fi
\ifx\subparagraph\undefined\else
  \let\oldsubparagraph\subparagraph
  \renewcommand{\subparagraph}{
    \@ifstar
      \xxxSubParagraphStar
      \xxxSubParagraphNoStar
  }
  \newcommand{\xxxSubParagraphStar}[1]{\oldsubparagraph*{#1}\mbox{}}
  \newcommand{\xxxSubParagraphNoStar}[1]{\oldsubparagraph{#1}\mbox{}}
\fi
\makeatother


\providecommand{\tightlist}{%
  \setlength{\itemsep}{0pt}\setlength{\parskip}{0pt}}\usepackage{longtable,booktabs,array}
\usepackage{calc} % for calculating minipage widths
% Correct order of tables after \paragraph or \subparagraph
\usepackage{etoolbox}
\makeatletter
\patchcmd\longtable{\par}{\if@noskipsec\mbox{}\fi\par}{}{}
\makeatother
% Allow footnotes in longtable head/foot
\IfFileExists{footnotehyper.sty}{\usepackage{footnotehyper}}{\usepackage{footnote}}
\makesavenoteenv{longtable}
\usepackage{graphicx}
\makeatletter
\newsavebox\pandoc@box
\newcommand*\pandocbounded[1]{% scales image to fit in text height/width
  \sbox\pandoc@box{#1}%
  \Gscale@div\@tempa{\textheight}{\dimexpr\ht\pandoc@box+\dp\pandoc@box\relax}%
  \Gscale@div\@tempb{\linewidth}{\wd\pandoc@box}%
  \ifdim\@tempb\p@<\@tempa\p@\let\@tempa\@tempb\fi% select the smaller of both
  \ifdim\@tempa\p@<\p@\scalebox{\@tempa}{\usebox\pandoc@box}%
  \else\usebox{\pandoc@box}%
  \fi%
}
% Set default figure placement to htbp
\def\fps@figure{htbp}
\makeatother

\usepackage{paracol}
\usepackage{paracol}
\newenvironment{cheatboxparacol}{\par\vspace{2pt}\noindent\begin{tcolorbox}[enhanced, sharp corners, colback=gray!5]}{\end{tcolorbox}\vspace{2pt}}

\newcommand{\StartCheatColumns}[1]{\begin{paracol}{#1}}
\newcommand{\EndCheatColumns}{\end{paracol}}

\usepackage{tikz}
\usepackage{etoolbox}
\usepackage[most]{tcolorbox}
% \usepackage[margin=1.5cm]{geometry}
\makeatletter
\@ifpackageloaded{caption}{}{\usepackage{caption}}
\AtBeginDocument{%
\ifdefined\contentsname
  \renewcommand*\contentsname{Table of contents}
\else
  \newcommand\contentsname{Table of contents}
\fi
\ifdefined\listfigurename
  \renewcommand*\listfigurename{List of Figures}
\else
  \newcommand\listfigurename{List of Figures}
\fi
\ifdefined\listtablename
  \renewcommand*\listtablename{List of Tables}
\else
  \newcommand\listtablename{List of Tables}
\fi
\ifdefined\figurename
  \renewcommand*\figurename{Figure}
\else
  \newcommand\figurename{Figure}
\fi
\ifdefined\tablename
  \renewcommand*\tablename{Table}
\else
  \newcommand\tablename{Table}
\fi
}
\@ifpackageloaded{float}{}{\usepackage{float}}
\floatstyle{ruled}
\@ifundefined{c@chapter}{\newfloat{codelisting}{h}{lop}}{\newfloat{codelisting}{h}{lop}[chapter]}
\floatname{codelisting}{Listing}
\newcommand*\listoflistings{\listof{codelisting}{List of Listings}}
\makeatother
\makeatletter
\makeatother
\makeatletter
\@ifpackageloaded{caption}{}{\usepackage{caption}}
\@ifpackageloaded{subcaption}{}{\usepackage{subcaption}}
\makeatother
\makeatletter
\@ifpackageloaded{tcolorbox}{}{\usepackage[skins,breakable]{tcolorbox}}
\makeatother
\makeatletter
\@ifundefined{shadecolor}{\definecolor{shadecolor}{rgb}{.97, .97, .97}}{}
\makeatother
\makeatletter
\makeatother
\makeatletter
\ifdefined\Shaded\renewenvironment{Shaded}{\begin{tcolorbox}[frame hidden, breakable, sharp corners, boxrule=0pt, interior hidden, enhanced, borderline west={3pt}{0pt}{shadecolor}]}{\end{tcolorbox}}\fi
\makeatother

\usepackage{bookmark}

\IfFileExists{xurl.sty}{\usepackage{xurl}}{} % add URL line breaks if available
\urlstyle{same} % disable monospaced font for URLs
\hypersetup{
  colorlinks=true,
  linkcolor={blue},
  filecolor={Maroon},
  citecolor={Blue},
  urlcolor={Blue},
  pdfcreator={LaTeX via pandoc}}


\author{}
\date{}

\begin{document}
\newcounter{colnum}
\setcounter{colnum}{1}
\newlength{\colheight}
\setlength{\colheight}{0pt}

\def\StartCheatColumns#1{%
  \begin{paracol}{#1}%
    \setcounter{colnum}{1}%
  \setlength{\colheight}{0pt}%
  }
  
  \def\EndCheatColumns{%
  \end{paracol}
}

\StartCheatColumns{3} % default; overridden by Lua

% Custom font size for body text

% Custom page geometry


\begin{cheatboxparacol}

Rekursion ist eine Technik, bei der eine Funktion sich selbst aufruft.

Beispiel in Python:

\begin{verbatim}
def factorial(n):
    return 1 if n == 0 else n * factorial(n-1)
\end{verbatim}

\end{cheatboxparacol}

\begin{cheatboxparacol}

Listen, Mengen und Dictionaries sind grundlegende Datenstrukturen.

\begin{itemize}
\tightlist
\item
  Liste: \texttt{{[}1,\ 2,\ 3{]}}
\item
  Set: \texttt{\{1,\ 2,\ 3\}}
\item
  Dictionary: \texttt{\{"a":\ 1,\ "b":\ 2\}}
\end{itemize}

\end{cheatboxparacol}

\begin{cheatboxparacol}

Anonyme Funktionen:

\begin{verbatim}
f = lambda x: x * x
print(f(4))  # 16
\end{verbatim}

\end{cheatboxparacol}

\begin{cheatboxparacol}

Rekursion ist eine Technik, bei der eine Funktion sich selbst aufruft.

Beispiel in Python:

\begin{verbatim}
def factorial(n):
    return 1 if n == 0 else n * factorial(n-1)
\end{verbatim}

\end{cheatboxparacol}

\begin{cheatboxparacol}

Listen, Mengen und Dictionaries sind grundlegende Datenstrukturen.

\begin{itemize}
\tightlist
\item
  Liste: \texttt{{[}1,\ 2,\ 3{]}}
\item
  Set: \texttt{\{1,\ 2,\ 3\}}
\item
  Dictionary: \texttt{\{"a":\ 1,\ "b":\ 2\}}
\end{itemize}

\end{cheatboxparacol}

\begin{cheatboxparacol}

Anonyme Funktionen:

\begin{verbatim}
f = lambda x: x * x
print(f(4))  # 16
\end{verbatim}

\end{cheatboxparacol}

\begin{cheatboxparacol}

Rekursion ist eine Technik, bei der eine Funktion sich selbst aufruft.

Beispiel in Python:

\begin{verbatim}
def factorial(n):
    return 1 if n == 0 else n * factorial(n-1)
\end{verbatim}

\end{cheatboxparacol}

\begin{cheatboxparacol}

Listen, Mengen und Dictionaries sind grundlegende Datenstrukturen.

\begin{itemize}
\tightlist
\item
  Liste: \texttt{{[}1,\ 2,\ 3{]}}
\item
  Set: \texttt{\{1,\ 2,\ 3\}}
\item
  Dictionary: \texttt{\{"a":\ 1,\ "b":\ 2\}}
\end{itemize}

\end{cheatboxparacol}

\begin{cheatboxparacol}

Anonyme Funktionen:

\begin{verbatim}
f = lambda x: x * x
print(f(4))  # 16
\end{verbatim}

\end{cheatboxparacol}

\begin{cheatboxparacol}

Rekursion ist eine Technik, bei der eine Funktion sich selbst aufruft.

Beispiel in Python:

\begin{verbatim}
def factorial(n):
    return 1 if n == 0 else n * factorial(n-1)
\end{verbatim}

\end{cheatboxparacol}

\begin{cheatboxparacol}

Listen, Mengen und Dictionaries sind grundlegende Datenstrukturen.

\begin{itemize}
\tightlist
\item
  Liste: \texttt{{[}1,\ 2,\ 3{]}}
\item
  Set: \texttt{\{1,\ 2,\ 3\}}
\item
  Dictionary: \texttt{\{"a":\ 1,\ "b":\ 2\}}
\end{itemize}

\end{cheatboxparacol}

\begin{cheatboxparacol}

Anonyme Funktionen:

\begin{verbatim}
f = lambda x: x * x
print(f(4))  # 16
\end{verbatim}

\end{cheatboxparacol}

\begin{cheatboxparacol}

Rekursion ist eine Technik, bei der eine Funktion sich selbst aufruft.

Beispiel in Python:

\begin{verbatim}
def factorial(n):
    return 1 if n == 0 else n * factorial(n-1)
\end{verbatim}

\end{cheatboxparacol}

\begin{cheatboxparacol}

Listen, Mengen und Dictionaries sind grundlegende Datenstrukturen.

\begin{itemize}
\tightlist
\item
  Liste: \texttt{{[}1,\ 2,\ 3{]}}
\item
  Set: \texttt{\{1,\ 2,\ 3\}}
\item
  Dictionary: \texttt{\{"a":\ 1,\ "b":\ 2\}}
\end{itemize}

\end{cheatboxparacol}

\begin{cheatboxparacol}

Anonyme Funktionen:

\begin{verbatim}
f = lambda x: x * x
print(f(4))  # 16
\end{verbatim}

\end{cheatboxparacol}

\newpage 
\newpage 
\newpage

\newpage{}

\newpage{}

\newpage{}

\begin{cheatboxparacol}

Rekursion ist eine Technik, bei der eine Funktion sich selbst aufruft.

Beispiel in Python:

\begin{verbatim}
def factorial(n):
    return 1 if n == 0 else n * factorial(n-1)
\end{verbatim}

\end{cheatboxparacol}

\begin{cheatboxparacol}

Listen, Mengen und Dictionaries sind grundlegende Datenstrukturen.

\begin{itemize}
\tightlist
\item
  Liste: \texttt{{[}1,\ 2,\ 3{]}}
\item
  Set: \texttt{\{1,\ 2,\ 3\}}
\item
  Dictionary: \texttt{\{"a":\ 1,\ "b":\ 2\}}
\end{itemize}

\end{cheatboxparacol}

\begin{cheatboxparacol}

Anonyme Funktionen:

\begin{verbatim}
f = lambda x: x * x
print(f(4))  # 16
\end{verbatim}

\end{cheatboxparacol}

\begin{cheatboxparacol}

Rekursion ist eine Technik, bei der eine Funktion sich selbst aufruft.

Beispiel in Python:

\begin{verbatim}
def factorial(n):
    return 1 if n == 0 else n * factorial(n-1)
\end{verbatim}

\end{cheatboxparacol}

\begin{cheatboxparacol}

Listen, Mengen und Dictionaries sind grundlegende Datenstrukturen.

\begin{itemize}
\tightlist
\item
  Liste: \texttt{{[}1,\ 2,\ 3{]}}
\item
  Set: \texttt{\{1,\ 2,\ 3\}}
\item
  Dictionary: \texttt{\{"a":\ 1,\ "b":\ 2\}}
\end{itemize}

\end{cheatboxparacol}

\begin{cheatboxparacol}

Anonyme Funktionen:

\begin{verbatim}
f = lambda x: x * x
print(f(4))  # 16
\end{verbatim}

\end{cheatboxparacol}

\begin{cheatboxparacol}

Rekursion ist eine Technik, bei der eine Funktion sich selbst aufruft.

Beispiel in Python:

\begin{verbatim}
def factorial(n):
    return 1 if n == 0 else n * factorial(n-1)
\end{verbatim}

\end{cheatboxparacol}

\begin{cheatboxparacol}

Listen, Mengen und Dictionaries sind grundlegende Datenstrukturen.

\begin{itemize}
\tightlist
\item
  Liste: \texttt{{[}1,\ 2,\ 3{]}}
\item
  Set: \texttt{\{1,\ 2,\ 3\}}
\item
  Dictionary: \texttt{\{"a":\ 1,\ "b":\ 2\}}
\end{itemize}

\end{cheatboxparacol}

\begin{cheatboxparacol}

Anonyme Funktionen:

\begin{verbatim}
f = lambda x: x * x
print(f(4))  # 16
\end{verbatim}

\end{cheatboxparacol}

\begin{cheatboxparacol}

Rekursion ist eine Technik, bei der eine Funktion sich selbst aufruft.

Beispiel in Python:

\begin{verbatim}
def factorial(n):
    return 1 if n == 0 else n * factorial(n-1)
\end{verbatim}

\end{cheatboxparacol}

\begin{cheatboxparacol}

Listen, Mengen und Dictionaries sind grundlegende Datenstrukturen.

\begin{itemize}
\tightlist
\item
  Liste: \texttt{{[}1,\ 2,\ 3{]}}
\item
  Set: \texttt{\{1,\ 2,\ 3\}}
\item
  Dictionary: \texttt{\{"a":\ 1,\ "b":\ 2\}}
\end{itemize}

\end{cheatboxparacol}

\begin{cheatboxparacol}

Anonyme Funktionen:

\begin{verbatim}
f = lambda x: x * x
print(f(4))  # 16
\end{verbatim}

\end{cheatboxparacol}

\begin{cheatboxparacol}

Rekursion ist eine Technik, bei der eine Funktion sich selbst aufruft.

Beispiel in Python:

\begin{verbatim}
def factorial(n):
    return 1 if n == 0 else n * factorial(n-1)
\end{verbatim}

\end{cheatboxparacol}

\begin{cheatboxparacol}

Listen, Mengen und Dictionaries sind grundlegende Datenstrukturen.

\begin{itemize}
\tightlist
\item
  Liste: \texttt{{[}1,\ 2,\ 3{]}}
\item
  Set: \texttt{\{1,\ 2,\ 3\}}
\item
  Dictionary: \texttt{\{"a":\ 1,\ "b":\ 2\}}
\end{itemize}

\end{cheatboxparacol}

\begin{cheatboxparacol}

Anonyme Funktionen:

\begin{verbatim}
f = lambda x: x * x
print(f(4))  # 16
\end{verbatim}

\end{cheatboxparacol}



\EndCheatColumns
\end{document}
